% Options for packages loaded elsewhere
\PassOptionsToPackage{unicode}{hyperref}
\PassOptionsToPackage{hyphens}{url}
%
\documentclass[A4paper
]{article}
\title{Guide \LaTeX}
\author{Rémi VAN BOXEM}

\usepackage[french]{babel}
\usepackage{amsmath,amssymb}
\usepackage{lmodern}
\usepackage{iftex}
\ifPDFTeX
  \usepackage[T1]{fontenc}
  \usepackage[utf8]{inputenc}
  \usepackage{textcomp} % provide euro and other symbols
\else % if luatex or xetex
  \usepackage{unicode-math}
  \defaultfontfeatures{Scale=MatchLowercase}
  \defaultfontfeatures[\rmfamily]{Ligatures=TeX,Scale=1}
\fi
% Use upquote if available, for straight quotes in verbatim environments
\IfFileExists{upquote.sty}{\usepackage{upquote}}{}
\IfFileExists{microtype.sty}{% use microtype if available
  \usepackage[]{microtype}
  \UseMicrotypeSet[protrusion]{basicmath} % disable protrusion for tt fonts
}{}
\makeatletter
\@ifundefined{KOMAClassName}{% if non-KOMA class
  \IfFileExists{parskip.sty}{%
    \usepackage{parskip}
  }{% else
    \setlength{\parindent}{0pt}
    \setlength{\parskip}{6pt plus 2pt minus 1pt}}
}{% if KOMA class
  \KOMAoptions{parskip=half}}
\makeatother
\usepackage{xcolor}
\IfFileExists{xurl.sty}{\usepackage{xurl}}{} % add URL line breaks if available
\IfFileExists{bookmark.sty}{\usepackage{bookmark}}{\usepackage{hyperref}}
\hypersetup{
  pdftitle={Guide LaTeX},
  pdfauthor={Rémi VAN BOXEM},
  hidelinks,
  pdfcreator={Junia}}
\urlstyle{same} % disable monospaced font for URLs
\usepackage{color}
\usepackage{fancyvrb}
\newcommand{\VerbBar}{|}
\newcommand{\VERB}{\Verb[commandchars=\\\{\}]}
\DefineVerbatimEnvironment{Highlighting}{Verbatim}{commandchars=\\\{\}}
% Add ',fontsize=\small' for more characters per line
\newenvironment{Shaded}{}{}
\newcommand{\AlertTok}[1]{\textcolor[rgb]{1.00,0.00,0.00}{\textbf{#1}}}
\newcommand{\AnnotationTok}[1]{\textcolor[rgb]{0.38,0.63,0.69}{\textbf{\textit{#1}}}}
\newcommand{\AttributeTok}[1]{\textcolor[rgb]{0.49,0.56,0.16}{#1}}
\newcommand{\BaseNTok}[1]{\textcolor[rgb]{0.25,0.63,0.44}{#1}}
\newcommand{\BuiltInTok}[1]{#1}
\newcommand{\CharTok}[1]{\textcolor[rgb]{0.25,0.44,0.63}{#1}}
\newcommand{\CommentTok}[1]{\textcolor[rgb]{0.38,0.63,0.69}{\textit{#1}}}
\newcommand{\CommentVarTok}[1]{\textcolor[rgb]{0.38,0.63,0.69}{\textbf{\textit{#1}}}}
\newcommand{\ConstantTok}[1]{\textcolor[rgb]{0.53,0.00,0.00}{#1}}
\newcommand{\ControlFlowTok}[1]{\textcolor[rgb]{0.00,0.44,0.13}{\textbf{#1}}}
\newcommand{\DataTypeTok}[1]{\textcolor[rgb]{0.56,0.13,0.00}{#1}}
\newcommand{\DecValTok}[1]{\textcolor[rgb]{0.25,0.63,0.44}{#1}}
\newcommand{\DocumentationTok}[1]{\textcolor[rgb]{0.73,0.13,0.13}{\textit{#1}}}
\newcommand{\ErrorTok}[1]{\textcolor[rgb]{1.00,0.00,0.00}{\textbf{#1}}}
\newcommand{\ExtensionTok}[1]{#1}
\newcommand{\FloatTok}[1]{\textcolor[rgb]{0.25,0.63,0.44}{#1}}
\newcommand{\FunctionTok}[1]{\textcolor[rgb]{0.02,0.16,0.49}{#1}}
\newcommand{\ImportTok}[1]{#1}
\newcommand{\InformationTok}[1]{\textcolor[rgb]{0.38,0.63,0.69}{\textbf{\textit{#1}}}}
\newcommand{\KeywordTok}[1]{\textcolor[rgb]{0.00,0.44,0.13}{\textbf{#1}}}
\newcommand{\NormalTok}[1]{#1}
\newcommand{\OperatorTok}[1]{\textcolor[rgb]{0.40,0.40,0.40}{#1}}
\newcommand{\OtherTok}[1]{\textcolor[rgb]{0.00,0.44,0.13}{#1}}
\newcommand{\PreprocessorTok}[1]{\textcolor[rgb]{0.74,0.48,0.00}{#1}}
\newcommand{\RegionMarkerTok}[1]{#1}
\newcommand{\SpecialCharTok}[1]{\textcolor[rgb]{0.25,0.44,0.63}{#1}}
\newcommand{\SpecialStringTok}[1]{\textcolor[rgb]{0.73,0.40,0.53}{#1}}
\newcommand{\StringTok}[1]{\textcolor[rgb]{0.25,0.44,0.63}{#1}}
\newcommand{\VariableTok}[1]{\textcolor[rgb]{0.10,0.09,0.49}{#1}}
\newcommand{\VerbatimStringTok}[1]{\textcolor[rgb]{0.25,0.44,0.63}{#1}}
\newcommand{\WarningTok}[1]{\textcolor[rgb]{0.38,0.63,0.69}{\textbf{\textit{#1}}}}
\usepackage{longtable,booktabs,array}
\usepackage{calc} % for calculating minipage widths
% Correct order of tables after \paragraph or \subparagraph
\usepackage{etoolbox}
\makeatletter
\patchcmd\longtable{\par}{\if@noskipsec\mbox{}\fi\par}{}{}
\makeatother
% Allow footnotes in longtable head/foot
\IfFileExists{footnotehyper.sty}{\usepackage{footnotehyper}}{\usepackage{footnote}}
\makesavenoteenv{longtable}
\setlength{\emergencystretch}{3em} % prevent overfull lines
\providecommand{\tightlist}{%
  \setlength{\itemsep}{0pt}\setlength{\parskip}{0pt}}
\setcounter{secnumdepth}{5}
\ifLuaTeX
  \usepackage{selnolig}  % disable illegal ligatures
\fi

\begin{document}
\maketitle
\begin{abstract}
Ceci est un guide basique du \LaTeX. Réalisez aisément des documents
simples et plus propre qu'un logiciel WYSIWYG. Ce guide est fait pour
être lu à côté d'un environnement \LaTeX opérationnel.
\end{abstract}

{
\setcounter{tocdepth}{3}
\tableofcontents
}
\hypertarget{introduction}{%
\section{Introduction}\label{introduction}}

Dans cette section, vous allez pouvoir choisir comment souhaitez-vous
rendre votre document \LaTeX. Il faut savoir que \LaTeX est un langage
compilé, donc il faudra au minima un compilateur \LaTeX et un éditeur. Il
existe toutefois une solution sans aucune installation en utilisant
simplement votre navigateur.

\hypertarget{en-ligne}{%
\subsection{En ligne}\label{en-ligne}}

Une solution 100\% en ligne est d'utiliser le site
\textbf{\href{https://www.overleaf.com/}{Overleaf}}. Vous pouvez vous
créer gratuitement un compte sur Overleaf. Une fois sur votre espace,
cliquez sur le bouton vert \textbf{New Project} puis sur \textbf{Blank
Project}. Choississez un nom pour votre projet et on est parti !

\hypertarget{sur-ma-machine}{%
\subsection{Sur ma machine}\label{sur-ma-machine}}

Dans cette sous-section, on va vous installer les outils afin de
compiler localement votre fichier. Référez-vous à la partie
correspondante à votre système d'exploitation. Dans ce guide (qui est
prévu pour les ISEN, bonjour à vous), nous allons utiiser Visual Studio
Code comme éditeur de texte (avec l'excellente extension \LaTeX
Workshop).

Pour installer Visual Studio Code, référez-vous au guide d'installation
\href{https://code.visualstudio.com/docs/setup/}{suivant}.

\hypertarget{macos}{%
\subsubsection{macOS}\label{macos}}

La distribution \LaTeX recommandée est
\href{https://tug.org/mactex/}{MacTeX}, les détails d'installations sont
présents sur la page.

\hypertarget{windows}{%
\subsubsection{Windows}\label{windows}}

La distribution \LaTeX recommandée pour Windows est
\href{https://miktex.org}{MiKTeX}. Vous devez également installer un
interpréteur Perl, ma recommendation est
\href{https://strawberryperl.com}{Strawberry Perl}. C'est un
environnement perl pour MS Windows contenant tout ce dont vous avez
besoin pour exécuter et développer des applications perl. Il est conçu
pour être aussi proche que possible de l'environnement perl des systèmes
UNIX.

Redémarrez votre ordinateur. Il n'est pas rare que le compilateur \LaTeX
MiKTeX soit capricieux sur Windows et nécessite le redémarrage de
l'ordinateur pour être correctement inclus dans le \texttt{PATH}.

Ensuite, lors de votre première compilation, il est possible que votre
document ne compile juste pas. Soyez à l'affut d'une fenêtre de MiKTeX
vous demandant d'installer des paquets.

\hypertarget{linux}{%
\subsubsection{Linux}\label{linux}}

Téléchargez la distribution \textbf{\TeX Live} depuis votre package
manager :

\begin{longtable}[]{@{}ll@{}}
\toprule
Distribution & Nom du paquet \\
\midrule
\endhead
Arch Linux & \texttt{texlive-most} \\
Debian & \texttt{texlive-full} \\
Fedora & \texttt{texlive-scheme-medium} \\
Gentoo & \texttt{app-text/texlive} \\
NixOS & \texttt{nixpkgs.texlive.combined.scheme-medium} \\
openSUSE & \texttt{texlive-latex} \\
Ubuntu & \texttt{texlive-full} \\
\bottomrule
\end{longtable}

Une fois \TeX Live d'installé, vous n'avez plus rien à faire. Vous êtes
opérationnel !

\hypertarget{comme-le-bigorneau-adhuxe8re-au-bitume}{%
\section{Comme le bigorneau adhère au
bitume}\label{comme-le-bigorneau-adhuxe8re-au-bitume}}

Un drôle de titre, non ? Le principe de cette partie est de poser les
premières bases, sans rentrer dans le complexe pour créer un document
\LaTeX. Vous retrouverez en annexes les différents cas particulier qui
seront prochainement annoncés.

\hypertarget{votre-premier-document}{%
\subsection{Votre premier document}\label{votre-premier-document}}

C'est officiel, vous allez écrire votre premier document \LaTeX ! Sortez
votre meilleure plume, créez un fichier \texttt{main.tex}, ouvrez le
répertoire le contenant avec Visual Studio Code et écrivez ce qui suit
et compilez votre document. \emph{(Ces lignes vous seront expliqués
juste après.)}

\begin{Shaded}
\begin{Highlighting}[]
\BuiltInTok{\textbackslash{}documentclass}\NormalTok{\{}\ExtensionTok{article}\NormalTok{\}}
\KeywordTok{\textbackslash{}begin}\NormalTok{\{}\ExtensionTok{document}\NormalTok{\}}
\NormalTok{Hello World }
\KeywordTok{\textbackslash{}end}\NormalTok{\{}\ExtensionTok{document}\NormalTok{\}}
\end{Highlighting}
\end{Shaded}

Vous devriez voir une belle page blanche avec en haut à gauche ``Hello
World''. Si c'est le cas félicitations ! Si ce n'est pas le cas,
référez-vous aux procédures d'installation complètes en annexes.

La ligne \texttt{\textbackslash{}documentclass\{article\}} présise à
votre compilateur quel type de document vous écrivez actuellement. Il en
existe un certain nombre de base : \texttt{book} un format adapté à la
rédaction de gros ouvrages (plusieurs centaines de pages),
\texttt{report} un format adapté pour des documents de taille modéré et
la classe \texttt{article} pour des documents extrêmement courts.

Ces trois formats sont les formats natifs que vous risquez de rencontrer
le plus souvent.

\begin{quote}
Il existe aussi d'autres formats plus exotiques tel que \texttt{beamer}
une classe permettant de faire des présentations, \texttt{letter} une
classe spéciale pour la rédaction de courrier dans les normes
américaines et \texttt{lettre} la même chose mais avec les standards
européens établis par l'Observatoire de Genève.
\end{quote}

Cette ligne doit être impérativement la première de votre fichier
\texttt{.tex}.

Ensuite vous avez le \texttt{\textbackslash{}begin\{document\}} et le
\texttt{\textbackslash{}end\{document\}}. En écrivant ceci vous dites
que tout ce qui est compris entre ces deux balises est dans le document
(visuellement parlant). On parle d'environnement de document.

Et enfin, notre ``Hello World'' est affiché puisqu'il dans
l'environnement de document.

\hypertarget{document-oui-mais}{%
\subsection{Document oui, mais\ldots{}}\label{document-oui-mais}}

\hypertarget{muxe9tadonnuxe9es}{%
\subsubsection{Métadonnées}\label{muxe9tadonnuxe9es}}

En \LaTeX, vous avez un ensemble de métadonnées simples : le titre,
l'auteur et la date de rédaction/publication du document. \emph{Selon la
classe de document choisie, certaines métadonnées supplémentaires
peuvent être demandées afin de personnaliser au mieux le document.}

Pour définir ces métadonnées, vous pouvez écrire en haut de votre
document :

\begin{Shaded}
\begin{Highlighting}[]
\FunctionTok{\textbackslash{}title}\NormalTok{\{Mon super titre\}}
\FunctionTok{\textbackslash{}author}\NormalTok{\{C\textquotesingle{}est moi !\}}
\end{Highlighting}
\end{Shaded}

Dans votre environnement de document, ajoutez à la première ligne :

\begin{Shaded}
\begin{Highlighting}[]
\FunctionTok{\textbackslash{}maketitle}
\end{Highlighting}
\end{Shaded}

Compilez votre document et pouf ! Vous avez un joli titre tout propre !
Mais\ldots{} La date du jour est en anglais ? Corrigons cela !

\hypertarget{localisation-du-document}{%
\subsubsection{Localisation du
document}\label{localisation-du-document}}

\LaTeX est prévu pour être utilisé dans le monde entier, c'est pour cela
qu'on peut préciser la langue et les normes locales qu'on souhaite
utiliser.

Pour choisir cela, juste après votre définition de classe de document,
ajoutez :

\begin{Shaded}
\begin{Highlighting}[]
\BuiltInTok{\textbackslash{}usepackage}\NormalTok{[french]\{}\ExtensionTok{babel}\NormalTok{\} }\CommentTok{\% Normes et traductions}
\BuiltInTok{\textbackslash{}usepackage}\NormalTok{[T1]\{}\ExtensionTok{fontenc}\NormalTok{\} }\CommentTok{\% Usage des caractères spéciaux}
\end{Highlighting}
\end{Shaded}

La première ligne importe le paquet \texttt{babel} et lui donne comme
option qu'on utilise la langue \textbf{\emph{française}} pour notre
document. Cela va permettre de traduire ``Table of Contents'' par
``Table des matières'', ``Abstract'' par ``Résumé''\ldots{} et utiliser
les normes françaises, notamment pour les dates.

Compilez votre document et découvrez une page de titre bien propre !

\hypertarget{section-sous-section-sous-sous-section}{%
\subsubsection{Section, sous-section,
sous-sous-section\ldots{}}\label{section-sous-section-sous-sous-section}}

Vous pouvez hiérachiser vos informations en faisant des parties. Les
trois les plus fréquentes sont \texttt{\textbackslash{}section}
\texttt{\textbackslash{}subsection} et
\texttt{\textbackslash{}subsubsection}. \emph{Il existe également
\texttt{\textbackslash{}part} \texttt{\textbackslash{}chapter} mais
nécessite la classe de document \texttt{book}.} Si ce n'est pas
suffisant vous avez aussi des \texttt{\textbackslash{}paragraph} et des
\texttt{\textbackslash{}subparagraph}.

Ces instructions possèdent tous la même syntaxe :
\texttt{\textbackslash{}typedesection\{titre\}}. Pour créer une section
que j'appelerai Junla, je peux écrire :

\begin{Shaded}
\begin{Highlighting}[]
\KeywordTok{\textbackslash{}section}\NormalTok{\{Junla\}}
\end{Highlighting}
\end{Shaded}

Pour faire un parallèle avec de la géométrie : La section correspond à
un parallélogramme, une sous section correspond à un rectangle ou à un
losange et une sous sous section peut être notre carré. Pour représenter
cela en \LaTeX, vous pouvez écrire :

\begin{Shaded}
\begin{Highlighting}[]
\KeywordTok{\textbackslash{}section}\NormalTok{\{Parallélogramme\}}
\NormalTok{En géométrie, un parallélogramme est un quadrilatère dont les segments diagonaux se coupent en leurs milieux.}

\KeywordTok{\textbackslash{}paragraph}\NormalTok{\{Propriété\}}
\NormalTok{Tout parallélogramme a un centre de symétrie : le point d\textquotesingle{}intersection de ses diagonales.}

\KeywordTok{\textbackslash{}subsection}\NormalTok{\{Losange\}}
\NormalTok{Un losange est un parallélogramme ayant au moins deux côtés consécutifs de même longueur. Il est même équilatéral.}

\KeywordTok{\textbackslash{}paragraph}\NormalTok{\{Propriété\} Les diagonales d\textquotesingle{}un losange sont les bissectrices de ses angles.}

\KeywordTok{\textbackslash{}subsection}\NormalTok{\{Rectangle\}}
\NormalTok{Un rectangle est un parallélogramme ayant au moins un angle droit. Il est même équiangle.}

\KeywordTok{\textbackslash{}subsubsection}\NormalTok{\{Carré\}}
\NormalTok{Un carré est un losange rectangle.}
\end{Highlighting}
\end{Shaded}

Compilez et vous avez trié par niveau les différents types de
quadrilatères particuliers.

Pour faire un joli sommaire, ajoutez après
\texttt{\textbackslash{}maketitle} :

\begin{Shaded}
\begin{Highlighting}[]
\FunctionTok{\textbackslash{}tableofcontents}
\end{Highlighting}
\end{Shaded}

En suivant la partie précédente, vous devez arriver à ce qui suit : Code
final :

\begin{Shaded}
\begin{Highlighting}[]
\BuiltInTok{\textbackslash{}documentclass}\NormalTok{\{}\ExtensionTok{article}\NormalTok{\}}
\BuiltInTok{\textbackslash{}usepackage}\NormalTok{[french]\{}\ExtensionTok{babel}\NormalTok{\}}
\BuiltInTok{\textbackslash{}usepackage}\NormalTok{[T1]\{}\ExtensionTok{fontenc}\NormalTok{\}}

\FunctionTok{\textbackslash{}title}\NormalTok{\{Mon super titre\}}
\FunctionTok{\textbackslash{}author}\NormalTok{\{C\textquotesingle{}est moi !\}}

\KeywordTok{\textbackslash{}begin}\NormalTok{\{}\ExtensionTok{document}\NormalTok{\}}
\FunctionTok{\textbackslash{}maketitle}
\FunctionTok{\textbackslash{}tableofcontents}
\NormalTok{Hello World }

\KeywordTok{\textbackslash{}section}\NormalTok{\{Parallélogramme\}}
\NormalTok{En géométrie, un parallélogramme est un quadrilatère dont les segments diagonaux se coupent en leurs milieux.}

\KeywordTok{\textbackslash{}paragraph}\NormalTok{\{Propriété\}}
\NormalTok{Tout parallélogramme a un centre de symétrie : le point d\textquotesingle{}intersection de ses diagonales.}

\KeywordTok{\textbackslash{}subsection}\NormalTok{\{Losange\}}
\NormalTok{Un losange est un parallélogramme ayant au moins deux côtés consécutifs de même longueur. Il est même équilatéral.}

\KeywordTok{\textbackslash{}paragraph}\NormalTok{\{Propriété\} Les diagonales d\textquotesingle{}un losange sont les bissectrices de ses angles.}

\KeywordTok{\textbackslash{}subsection}\NormalTok{\{Rectangle\}}
\NormalTok{Un rectangle est un parallélogramme ayant au moins un angle droit. Il est même équiangle.}

\KeywordTok{\textbackslash{}subsubsection}\NormalTok{\{Carré\}}
\NormalTok{Un carré est un losange rectangle.}

\KeywordTok{\textbackslash{}end}\NormalTok{\{}\ExtensionTok{document}\NormalTok{\}}
\end{Highlighting}
\end{Shaded}

\hypertarget{du-contenu-contenu-dans-un-beau-cadre}{%
\subsection{Du contenu contenu dans un beau
cadre}\label{du-contenu-contenu-dans-un-beau-cadre}}

Maintenant qu'on sait comment bien cadrer son contenu, on va apprendre à
écrire du contenu en \LaTeX.

\hypertarget{encadre-moi-uxe7a}{%
\subsubsection{Encadre-moi ça !}\label{encadre-moi-uxe7a}}

Ajouter des images, première chose qu'on fait avec un logiciel de
traitement de texte. On se bat avec le placement de cette dernière avant
d'abandonner et de sortir PowerPoint pour faire son CV.

On va voir comment en \LaTeX, on importe une image.

\begin{Shaded}
\begin{Highlighting}[]
\BuiltInTok{\textbackslash{}documentclass}\NormalTok{\{}\ExtensionTok{article}\NormalTok{\}}
\BuiltInTok{\textbackslash{}usepackage}\NormalTok{\{}\ExtensionTok{graphicx}\NormalTok{\}}
\FunctionTok{\textbackslash{}graphicspath}\NormalTok{\{ \{./images/\} \}}

\KeywordTok{\textbackslash{}begin}\NormalTok{\{}\ExtensionTok{document}\NormalTok{\}}
\NormalTok{L\textquotesingle{}univers est immense et il semble être homogène, }
\NormalTok{à grande échelle, partout où nous regardons.}

\BuiltInTok{\textbackslash{}includegraphics}\NormalTok{\{}\ExtensionTok{universe}\NormalTok{\}}

\NormalTok{Voici une photo de la galaxie juste au dessus.}
\KeywordTok{\textbackslash{}end}\NormalTok{\{}\ExtensionTok{document}\NormalTok{\}}
\end{Highlighting}
\end{Shaded}

Et pouf ! L'image est correctement importée ! \LaTeX ne peut pas gérer
les images par lui-même, nous devons donc utiliser le paquet
\texttt{graphicx}. Pour l'utiliser, nous incluons la ligne suivante dans
le préambule : \texttt{\textbackslash{}usepackage\{graphicx\}}.

Il se peut que l'image ne soit pas au bon format. On peut rajouter le
paramètre \texttt{scale} afin d'affiner la taille de l'image.

\begin{Shaded}
\begin{Highlighting}[]
\BuiltInTok{\textbackslash{}includegraphics}\NormalTok{[scale=1.5]\{}\ExtensionTok{universe}\NormalTok{\} }\CommentTok{\% On aggrandit l\textquotesingle{}image}
\end{Highlighting}
\end{Shaded}

\textbf{Note :} L'extension du fichier peut être incluse, mais c'est une
bonne idée de l'omettre. Si l'extension de fichier est omise, \LaTeX
recherchera tous les formats supportés.

Plus d'informations sur la gestion du positionnement des images juste
\href{https://www.overleaf.com/learn/latex/Inserting_Images}{ici}.

\hypertarget{des-uxe9quations}{%
\subsubsection{Des équations}\label{des-uxe9quations}}

Nous devons écrire une équation mathématiques très importante, mais vous
n'avez pas envie de scanner un bout de papier car il faut être sérieux !
\LaTeX à la solution pour vous !

\begin{verbatim}
$$E=mc^2$$
\end{verbatim}

\[E=mc²\]

Félicitations, vous venez de prouver la formule d'équivalence entre la
masse et l'énergie en relativité restreinte ! Voici un cookie.

Cette équation n'a toutefois rien de bien difficile à écrire, un simple
E = mc² aurait fait l'affaire. Toutefois, lorsqu'on a des équations plus
complexes comme : \(\binom{n}{k} = \frac{n!}{k!(n-k)!}\) ou encore
\(\lim_{h \rightarrow 0 } \frac{f(x+h)-f(x)}{h}\), on est bien content
d'avoir un logiciel externe pour !

En \LaTeX, pour écrire le paragraph précédent, on a du écrire :

\begin{Shaded}
\begin{Highlighting}[]
\NormalTok{Toutefois, lorsqu\textquotesingle{}on a des équations plus complexes  
comme : }\SpecialStringTok{$}\SpecialCharTok{\textbackslash{}binom}\SpecialStringTok{\{n\}\{k\} = 
}\SpecialCharTok{\textbackslash{}frac}
\SpecialStringTok{\{n!\}\{k!(n{-}k)!\}$
}\NormalTok{ 
ou encore
}\SpecialStringTok{$}\SpecialCharTok{\textbackslash{}lim}\SpecialStringTok{\_\{h }
\SpecialCharTok{\textbackslash{}rightarrow}\SpecialStringTok{ 0 \} }\SpecialCharTok{\textbackslash{}frac}
\SpecialStringTok{\{f(x+h){-}f(x)\}\{h\}$}\NormalTok{, on est bien content d\textquotesingle{}avoir un logiciel externe pour !}
\end{Highlighting}
\end{Shaded}

La différence entre le \texttt{\$math\$} et \texttt{\$\$math\$\$} est
que le premier va se contenter de se mettre là où vous l'avez placé
tandis que le double \texttt{\$\$} va centrer l'équation au centre de la
feuille.

Exemple :

\(\lim_{h \rightarrow 0 } \frac{f(x+h)-f(x)}{h}\)

\[\lim_{h \rightarrow 0 } \frac{f(x+h)-f(x)}{h}\]

\begin{Shaded}
\begin{Highlighting}[]
\SpecialStringTok{$}\SpecialCharTok{\textbackslash{}lim}\SpecialStringTok{\_\{h }\SpecialCharTok{\textbackslash{}rightarrow}\SpecialStringTok{ 0 \} }\SpecialCharTok{\textbackslash{}frac}\SpecialStringTok{\{f(x+h){-}f(x)\}\{h\}$}

\SpecialStringTok{$$}\SpecialCharTok{\textbackslash{}lim}\SpecialStringTok{\_\{h }\SpecialCharTok{\textbackslash{}rightarrow}\SpecialStringTok{ 0 \} }\SpecialCharTok{\textbackslash{}frac}\SpecialStringTok{\{f(x+h){-}f(x)\}\{h\}$$}
\end{Highlighting}
\end{Shaded}


\end{document}
